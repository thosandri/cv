
%%% Ceci est un commentaire  %%%%

\documentclass[10pt,a4paper,sans]{moderncv}   %%%% classe moderncv spécifique au cv
              %%% taille,format,police (sans ou roman)
              

\moderncvtheme[blue]{classic}
 %%%% couleur (red,green,blue,orange,purple,black)


  %%% paquet pour la gestion de la langue française 
\usepackage[utf8]{inputenc} %%% encodage
\usepackage[french]{babel} 
\usepackage[top=1.1cm, bottom=1.1cm, left=1.8cm, right=1.8cm]{geometry}
%s\usepackage[scale=0.88]{geometry}
%%%%  paquet pour gérer la dimensions des marges
\usepackage[french]{babel}

\setlength{\hintscolumnwidth}{2.8cm}
% Largeur de la colonne pour les dates

%%% infos personnelles
\firstname{Thomas}
\familyname{Sandri}
\title{Ingénieur Géomaticien}              
\address{11, bis bd Copernic}{77420 Champs-sur-Marne}
\email{tho.sandri@gmail.com}                      
\phone[mobile]{06 95 53 47 65}
%\phone[fixed]{03 26 21 09 27} telephone fixe
\extrainfo{26 ans -- Permis B}
\social[linkedin]{thomas.sandri}
%\social[linkedin]{thomas.sandri}
%\photo[64pt][0pt]{./test.jpg}
%%%% on commence le document %%%%
\begin{document}
\maketitle  %%%% pour créer les infos perso celles en haut à gauche


\section{Formation}

%%%% structure cventry : \cventry{dates}{intitulé par défaut en gras}{infos}{infos}{infos}{infos sur deuxième ligne}  
%\vspace{0.3cm}
\cventry{2014 -- 2015}{Mastère spécialisé PPMD}{Photogrammétrie, Positionnement et Mesures des Déformations}{ENSG}{}{Traitement du signal et des images, topométrie, photogrammétrie, 3D, génie logiciel}
\vspace{0.20cm} %%% pour espacement vertical
\cventry{2012 -- 2015}{Cycle ingénieur}{\'Ecole Nationale des Sciences Géographiques, ENSG}{Champs-sur-Marne}{}{Programmation, développement Web, base de données, systèmes d'information,
analyse numérique, géostatistiques, gestion de projet}
\vspace{0.20cm}
\cventry{2009 -- 2012}{Classe préparatoire aux grandes écoles}{Filière Physique et Technologie}{Lycée Etienne Oehmichen}{Châlons-en-champagne}{}
\vspace{0.1cm}
\cventry{2008 -- 2009}{Baccalauréat Scientifique}{Lycée Etienne Oehmichen}{Châlons-en-champagne}{}{}
\vspace{0.15cm}

\section{Expérience professionnelle}
%\vspace{0.3cm}
\cventry{Juin 2016\\à Aujourd'hui}{Ingénieur d'études}{Laboratoire de Recherche En Géodésie}{LAREG, Paris}{}{
\begin{itemize}
\item Étude et développement d'un prototype de filtrage pour l'assimilation de données multi-techniques.
\end{itemize}
}
\vspace{0.2cm}
\cventry{Novembre 2015\\à Février 2016}{Stage pluridisciplinaire}{Università Degli Studi di Udine}{Italie}{}{
\begin{itemize}
\item Développement d'une méthode de détection et de restauration des ombres dans les 
images aériennes très haute résolution en vue d'améliorer la classification de la végétation.
\end{itemize}
}
\vspace{0.2cm}
\cventry{Mai 2015\\à Octobre 2015}{Travail de fin d'études}{Institut de Mécanique Céleste et de Calcul des Éphémérides}{Paris}{}{
\begin{itemize}
\item Étude et implémentation du freinage atmosphérique dans un logiciel de propagation d'orbites;
\item Établissement d'un modèle statistique de prévisions de dates de rentrées atmosphériques de débris spatiaux.
\end{itemize}
}
\vspace{0.2cm}
\cventry{Mai 2013 \\ à Août 2013}{Stage terrain de l'ENSG}{Forcalquier}{France}{}{%\vspace{0.1cm}
\begin{itemize} %%% pour créer une liste
\item \'Etude des déformations d'un ouvrage d'art par auscultation;
\item Réalisation d'un géoïde local par nivellement et méthodes GNSS;
\item Modélisation 3D d'une chapelle par techniques photogrammétriques;
\item Classification de cultures à l'aide d'images RADAR.
\end{itemize}
}
\vspace{0.05cm}
\cventry{Juillet 2011 \\ à Août 2011}{Préparateur de commandes}{magasin E.Leclerc}{Saint-Martin-sur-le-Pré}{France}{}
\vspace{0.1cm}
\cventry{Juillet 2010 \\ à Août 2010}{Plongeur}{restaurant Courtepaille}{Châlons-en-champagne}{France}{}
\vspace{0.15cm}

\section{Compétences techniques}
%\vspace{0.3cm}
\cvdoubleitem{Environnement:}{Linux, Windows}{Programmation:}{C++, C, Fortran 90, Python, Bash}
\cvdoubleitem{Bibliothèques:}{Qt, OpenMP, Eigen, Numpy}{Logiciels:}{Matlab, Mathematica, Qgis}
\cvdoubleitem{Web:}{Html, PHP, JavaScript}{Outils:}{Git, Make, CMake, SQL}
%\cvitem{Environnement:}{Linux, Windows}
%\cvitem{Programmation:}{C/C++, Fortran 90, Python, OpenMP, MPI, Shell}
%\cvitem{Logiciels:}{Matlab, Mathematica}
%\cvitem{Web:}{HTML5, PHP, JavaScript}
%\cvitem{SIG:}{Qgis, ArcGIS}
%\cvitem{Bureautique:}{Office, \LaTeX}
%\cvitem{Outils:}{GCC, Make, Qt, Git, CUDA}

%\cvdoubleitem{Programmation:}{C++/C, Python, Fortran 90, Shell}{Web:}{HTML5, JavaScript, PHP}

%\vspace{0.15cm}

\vspace{0.15cm}

\section{Compétences linguistiques}
%\vspace{0.3cm}
\cvlanguage{Français:}{Langue maternelle}{}
\cvlanguage{Anglais:}{Courant - Toeic 820 / 990}{}
\cvlanguage{Italien:}{Courant}{}
\vspace{0.15cm}

\section{Centres d'intérêt}
%\vspace{0.3cm}
\cvitem{\textbf{Sports:}}{Basket-ball (Compétition - 11 ans), Football (Compétition - 5 ans), Course à pieds}

\end{document}

